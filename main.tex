\documentclass[a4paper,12pt,dutch]{article}
\usepackage{glossaries}
\usepackage[T1]{fontenc}
\usepackage{babel}
\usepackage{graphicx}
\usepackage[table,xcdraw]{xcolor}
\usepackage{hyperref}
\usepackage{blindtext}
\usepackage{geometry}
\usepackage{parskip}
\usepackage{mathtools}
\usepackage{siunitx}
\usepackage{listings}
\usepackage{csquotes}
\usepackage{caption}
\usepackage{subcaption}
\usepackage{comment}
\usepackage{pdfpages}
\usepackage{pict2e}
\usepackage{xcolor}

% \DeclareRobustCommand{\slashcirc}{{\mathpalette\doslashcirc\relax}}

% \makeatletter
% \newcommand\doslashcirc[2]{%
%   \sbox\z@{$#1\m@th\circ$}%
%   \setlength\unitlength{\wd\z@}
%   \begin{picture}(1,1)
%   \roundcap
%   \put(0,0){\box\z@}
%   \put(0,0){\line(1,1){1}}
%   \end{picture}%
% }
% \makeatother


% %% Some packages you will need
% \usepackage{pgfplots}
% \usepackage{pgfplotstable}
% \usepackage{booktabs}
% \usepackage{array}
% \usepackage{colortbl}


\definecolor{arduinoorange}{HTML}{FFA500}
\definecolor{arduinogray}{HTML}{808080}
\definecolor{arduinoblue}{HTML}{007ACC}
\definecolor{arduinogreen}{HTML}{469B00}

\lstset{
  language=C++,
  basicstyle=\ttfamily\footnotesize,
  keywordstyle=\color{arduinoorange},
  stringstyle=\color{arduinogreen},
  commentstyle=\color{arduinogray},
  moredelim=[s][\color{arduinoblue}]{\#}{\ },
  morekeywords={digitalRead,digitalWrite,pinMode,analogRead,analogWrite,Serial,begin,HIGH,LOW},
  frame=tb,
  tabsize=4,
  showstringspaces=false,
  breaklines=true,
  numbers=left,
  numberstyle=\tiny\color{arduinogray},
  numbersep=5pt,
  extendedchars=true,
  literate={á}{{\'a}}1 {ã}{{\~a}}1 {é}{{\'e}}1,
}

\lstdefinestyle{Arduino}
{
  language=C++,
  basicstyle=\ttfamily\footnotesize,
  keywordstyle=\color{arduinoorange},
  stringstyle=\color{arduinogreen},
  commentstyle=\color{arduinogray},
  moredelim=[s][\color{arduinoblue}]{\#}{\ },
  morekeywords={digitalRead,digitalWrite,pinMode,analogRead,analogWrite,Serial,begin},
  frame=tb,
  tabsize=4,
  showstringspaces=false,
  breaklines=true,
  numbers=left,
  numberstyle=\tiny\color{arduinogray},
  numbersep=5pt,
  extendedchars=true,
  literate={á}{{\'a}}1 {ã}{{\~a}}1 {é}{{\'e}}1,
  backgroundcolor=\color{black!85},
  rulecolor=\color{arduinoorange},
  frame=single,
  frameround=tttt,
  framexleftmargin=6pt,
  framexrightmargin=6pt,
  framextopmargin=6pt,
  framexbottommargin=6pt,
  breaklines=true,
  postbreak=\raisebox{0ex}[0ex][0ex]{\ensuremath{\color{red}\hookrightarrow\space}},
}

\usepackage[
    backend=biber,
    backref=true,
    backrefstyle=none,
    sortcites=true,
    sorting=none,
    doi=false, % doi informatie wordt niet weergegeven
    %uniquename=true,
    %uniquelist=true,
    maxcitenames=3,
    %issn=false, werkt niet
    language=american
]{biblatex}
\addbibresource{information/Sources.bib}
\DefineBibliographyStrings{dutch}{
    backrefpage = {blz.},
    backrefpages = {blz.},
}
\makeglossaries
\definecolor{Grey1}{HTML}{343434}
\graphicspath{{./Media/Figuren/}}
 \geometry{
 a4paper,
 total={170mm,257mm},
 left=20mm,
 top=20mm,
 }
\hypersetup{
    colorlinks=true,
    linkcolor=blue,
    filecolor=magenta,      
    urlcolor=cyan,
    pdftitle={Overleaf Example},
    pdfpagemode=FullScreen,
    }


\begin{document}
\title{
\includegraphics[width=3.5in]{img/logo.jpg} \\
\vspace*{1in}
\textbf{ELOMVO}\\
\textit{Prakticum}\\
Version 1
}
\author{
\vspace*{0.5in} \\
  Written by:\\
  Laurens van der Drift 21028605\\
  Tommy Dobos 22072411\\
  Jadon Wolfgang 24156809\\
\vspace*{0.2in} \\
    The Hague University of Applied Sciences\\
    \textbf{Electrical Engineering}\\
    Delft, The Netherlands
   } 
\maketitle
\phantomsection
\addcontentsline{toc}{section}{Introductie}
\textcolor{gray}{\textbf{Opmerking: alles wat in het grijs staat betreft de opdrachtvragen.}}
\include{main/TableOfContents}
\section{CASPOC Simulaties Boost}
\subsection{Inleiding en doel boostconverter}
\subsection{Simulatie}
\subsection{CASPOC Simulaties Buck-converter}
\subsubsection{Verschillende duty cycles}
\subsection{Continubedrijf en dutycycle}
\subsection{Inductieve belasting}
\subsection{Vermogen}
\subsection{Stroomwet van Kirchhof}
\subsection{Snelheid van de responsie}
\subsection{Conclusie}

\subsection{Simulatie}
\begin{figure}[h!]
    \centering
    \includegraphics[width=1\linewidth]{img/hfd1/Schema voor de simulatie van een boostconverter.png}
    \caption{Schema voor de simulatie van een boostconverter}
    \label{fig:Schema voor de simulatie van een boostconverter}
\end{figure}
\begin{table}[ht]
\centering
\resizebox{\textwidth}{!}{%
\begin{tabular}{|l|c|c|c|}
\hline
\textbf{Duty Cycle (\%)} & \textbf{Vuit (V) Rl=1MEG (onbelast)} & \textbf{Vuit (V) Rl=560 Ohm} & \textbf{Vuit (V) Rl=56 Ohm} \\ \hline
10 & 42.77 & 42.57 & 41.98 \\ \hline
20 & 48.41 & 48.01 & 47.26 \\ \hline
40 & 65.02 & 64.34 & 62.91 \\ \hline
50 & 77.95 & 77.23 & 75.25 \\ \hline
75 & 139.54 & 138.83 & 132.72 \\ \hline
\end{tabular}%
}
\caption{Uitgangsspanningen Vuit bij verschillende belasting en duty cycles}
\end{table}
Uit de simulaties blijkt dat de uitgangsspanning toeneemt met een hogere duty cycle, zoals verwacht bij een boostconverter. De belasting beïnvloedt de spanning: bij zwaardere belasting daalt de uitgangsspanning licht. De resultaten komen goed overeen met de theorie en tonen de correcte werking van de schakeling. De wet van Kirchhoff en \(UL = L \cdot \frac{dI}{dt}\) zijn duidelijk waarneembaar in de stroom- en spanningsgolfvormen.

\subsubsection{10\% dutycycle, 1 MEG ohm load}
\begin{figure}[ht]
    \centering
    \begin{subfigure}[b]{0.3\linewidth}
        \centering
        \includegraphics[width=\linewidth]{img/hfd1/hfd1-10pduty-1MEG-INDUCTOR.png}
        \caption{Spoelstroom}
        \label{fig:inductor}
    \end{subfigure}
    \hfill
    \begin{subfigure}[b]{0.3\linewidth}
        \centering
        \includegraphics[width=\linewidth]{img/hfd1/hfd1-10pduty-1MEG-DIODE.png}
        \caption{Diodestroom}
        \label{fig:diode}
    \end{subfigure}
    \hfill
    \begin{subfigure}[b]{0.3\linewidth}
        \centering
        \includegraphics[width=\linewidth]{img/hfd1/hfd1-10pduty-1MEG-DMOSFET.png}
        \caption{Mosfetstroom}
        \label{fig:mosfet}
    \end{subfigure}
    
    \caption{Vergelijking van stroom en spanning bij verschillende componenten met een weerstand van 1M\(\Omega\) en duty cycle van 10\%.}
    \label{fig:componenten}
\end{figure}




\subsubsection{40\% duty cycle, 560 ohm load}
\begin{figure}[h!]
    \centering
    \begin{subfigure}[b]{0.3\linewidth}
        \centering
        \includegraphics[width=\linewidth]{img/hfd1/hfd1-40pduty-560-INDUCTOR.png}
        \caption{Spoelstroom}
        \label{fig:inductor40}
    \end{subfigure}
    \hfill
    \begin{subfigure}[b]{0.3\linewidth}
        \centering
        \includegraphics[width=\linewidth]{img/hfd1/hfd1-40pduty-560-DIODE.png}
        \caption{Diodestroom}
        \label{fig:diode40}
    \end{subfigure}
    \hfill
    \begin{subfigure}[b]{0.3\linewidth}
        \centering
        \includegraphics[width=\linewidth]{img/hfd1/hfd1-40pduty-560-MOSFET.png}
        \caption{Mosfetstroom}
        \label{fig:mosfet40}
    \end{subfigure}
    
    \caption{Vergelijking van stroom en spanning bij verschillende componenten met een weerstand van 560\(\Omega\) en duty cycle van 40\%.}
    \label{fig:componenten40}
\end{figure}




\subsubsection{75\% duty cycle, 56 ohm load}

\begin{figure}[h!]
    \centering
    \begin{subfigure}[b]{0.3\linewidth}
        \centering
        \includegraphics[width=\linewidth]{img/hfd1/hfd1-75pduty-56-INDUCTOR.png}
        \caption{Spoelstroom}
        \label{fig:inductor75}
    \end{subfigure}
    \hfill
    \begin{subfigure}[b]{0.3\linewidth}
        \centering
        \includegraphics[width=\linewidth]{img/hfd1/hfd1-75pduty-56-DIODE.png}
        \caption{Diodestroom}
        \label{fig:diode75}
    \end{subfigure}
    \hfill
    \begin{subfigure}[b]{0.3\linewidth}
        \centering
        \includegraphics[width=\linewidth]{img/hfd1/hfd1-75pduty-56-DMOSFET.png}
        \caption{Mosfetstroom}
        \label{fig:mosfet75}
    \end{subfigure}
    
    \caption{Vergelijking van stroom en spanning bij verschillende componenten met een weerstand van 56\(\Omega\) en duty cycle van 75\%.}
    \label{fig:componenten75}
\end{figure}

De simulaties lieten zien dat de boostconverter zich gedraagt zoals verwacht. Het was duidelijk dat de uitgangsspanning stijgt als de duty cycle toeneemt, wat aantoont dat de converter goed werkt. De stroomwet van Kirchhoff is duidelijk te zien met de relatie tussen V en I. Kijkend naar de simulatie wordt ook \(U_L = L * \frac{dI}{dT}\) duidelijk. In de grafieken is ook de theorie van de lessen terug te zien. Zo is het effect van de duty cycle en de samenhang hiervan met de verschillende stromen goed terug te zien in de diode-, mosfet- en spoelstromen.
\subsection{CASPOC Simulaties Buck-converter}
\subsubsection{Verschillende duty cycles}

\subsection{Continubedrijf en dutycycle}
\begin{table}[h!]
\centering
\begin{tabular}{|c|c|}
\hline
\textbf{d} & \textbf{\(R_{load}\)} \\ \hline
10\% & 0 < \(R_{load}\) < 9\\ \hline
25\% & 0 < \(R_{load}\) < 11\\ \hline
50\% & 0 < \(R_{load}\) < 13\\ \hline
75\% & 0 < \(R_{load}\) < 15\\ \hline
90\% & 0 < \(R_{load}\) < 125\\ \hline
\end{tabular}
\caption{Waarde \(R_{load}\) per duty cycle voor continue blijven}
\label{tab:dutycycle_rload}
\end{table}
\subsection{Inductieve belasting}
\label{sub:1.5_ind_bel}
De schakeling is continu bij een waarde voor L = 125uH 

De parameters van tabel 1 met deze nieuwe waarde zijn te zien in tabel…  

Door de hogere spoel waarde zijn zowel de piek stroom en spanning, als de stationaire stroom en spanning een klein beetje gedaald. De bijbehorende tijden en de settling tijd zijn daarentegen toegenomen.  

\begin{table}[h!]
\centering
\resizebox{\textwidth}{!}{%
\begin{tabular}{|l|c|c|c|c|c|c|c|c|}
\hline
\textbf{d} & \textbf{Uitgangsspanning} & \textbf{Pieksapnning} & \textbf{Tijd piekspanning (s)} & \textbf{Stationaire spanning} & \textbf{Piekstroom} & \textbf{Tijd piekstroom (s)} & \textbf{Stationaire stroom} & \textbf{Settling tijd (s)} \\ \hline
10\% & 2 & 2,499 & 352\(\mu\) & 1,445 & 1,38 & 181\(\mu\) & 0,275 & 2,055m \\ \hline
\end{tabular}%
}
\caption{Nummerieke analyse buck converter 2}
\end{table}
\subsection{Vermogen}
\textcolor{gray}{Meet de gemiddelde ingangsstroom}\newline
Gemiddelde Ingangsstroom, \(I_{in, gem}\) = -0,623A , \(V_{in}\) = 20V , \(P_{in}\) = 12,46W \newline

\textcolor{gray}{Meet de gemiddelde uitgangsstroom}\newline
Gemiddelde Uitgangsstroom, \(I_{uit, gem}\) = -1,008A , \(V_{in}\) =  10,08V , \(P_{in}\) = 10,16W

\textcolor{gray}{Is het ingangsvermogen gelijk aan het uitgangsvermogen?}\newline
Het ingangs- en uitgangsvermogen zijn niet gelijk aan elkaar. Bij een ideale converter zou dit wel het geval zijn. In een echte buck converter zijn er echter kleine verliezen door weerstand in de spoel, het schakelen, en verliezen in de MOSFET en diode. Deze verliezen zorgen ervoor dat \(P_{in}\) iets groter is dan \(P_{uit}\).  
\subsection{Stroomwet van Kirchhof}
\textcolor{gray}{Meet de gemiddelde stroom door de spoel L}\newline
Gemiddelde spoel door de stroom, \(I_{L_{gem}}\) = 1,201A \newline
\newline
\textcolor{gray}{Meet de gemiddelde stroom door de Mosfet}\newline
Gemiddelde spoel door de MOSFET, \(I_{MOSFET_{gem}}\) = 0,623A\newline
\newline
\textcolor{gray}{Meet de gemiddelde stroom door de diode}\newline
Gemiddelde spoel door de Diode, \(I_{D_{gem}}\) = 0,534A\newline
\newline
\textcolor{gray}{Had je deze waarden ook uit de gemeten in en uitgangsstroom kunnen bepalen?}\newline
Nee deze waardes zijn niet te bepalen met de gemeten in en uitgangsstroom.


\subsection{Snelheid van de responsie}

\subsection{Conclusie}

Uit de simulaties bleek dat de boostconverter goed functioneerde zoals verwacht. Er was een duidelijke relatie tussen de duty cycle, de belasting en de uitgangsspanning. Dit toont de relatie van de converter aan met verschillende component waarden.

\section{Online Simulatie Buck}
\subsection{Buck converter}
Er zijn een aantal componentwaarden gegeven die we hebben ingevoerd in het simulatiecircuit. Deze waarden zijn weergegeven in \autoref{fig:Gegeven Componentwaarden}.
\begin{table}[h!]
\centering
\begin{tabular}{|l|l|}
\hline
\textbf{Component} & \textbf{Value} \\
\hline
Vin & 48 volt \\
L   & 220 $\mu$H \\
C   & 10 $\mu$F \\
Rout & 10 \\
Fs  & 50 kHz \\
d   & 25\% \\
\hline
\end{tabular}
\caption{Gegeven Componentwaarden}\label{fig:Gegeven Componentwaarden}
\end{table}
In de opdracht wordt aangegeven dat we deze componentwaarden moeten gebruiken om met verschillende waarden voor de duty cycle (D) te simuleren. De resultaten van deze simulaties zijn weergegeven in \autoref{fig:Overzicht van meetwaarden variabele duty cycle}
\begin{table}[h!]
\centering
\begin{tabular}{|c|c|c|c|c|c|c|c|}
\hline
\textbf{D} & \textbf{Vin} & \textbf{Iin} & \textbf{Vout} & \textbf{Iout} & \textbf{Pin} & \textbf{Pout} & \textbf{n} \\
\hline
10 & 48 & 0,559144 & 4,72 & 0,47204 & 26,838912 & 2,2280288 & 8,30\% \\
20 & 48 & 1,163 & 9,482 & 0,948209 & 55,824 & 8,990917738 & 16,11\% \\
30 & 48 & 1,807 & 14,191 & 1,419 & 86,736 & 20,137029 & 23,22\% \\
40 & 48 & 2,272 & 19,037 & 1,904 & 109,056 & 36,246448 & 33,24\% \\
50 & 48 & 2,365 & 23,625 & 2,362 & 113,52 & 55,80225 & 49,16\% \\
60 & 48 & 3,202 & 28,474 & 2,847 & 153,696 & 81,065478 & 52,74\% \\
70 & 48 & 3,355 & 33,314 & 3,331 & 161,04 & 110,968934 & 68,91\% \\
80 & 48 & 4,117 & 38,087 & 3,809 & 197,616 & 145,073383 & 73,41\% \\
90 & 48 & 4,463 & 43,001 & 4,3 & 214,224 & 184,9043 & 86,31\% \\
\hline
\end{tabular}
\caption{Overzicht van meetwaarden variabele duty cycle}\label{fig:Overzicht van meetwaarden variabele duty cycle}
\end{table}
Daarna wordt ons gevraagd om de waarden van Rout aan te passen. De resultaten voor de efficiëntie ($\eta$) bij verschillende waarden van Rout zijn weergegeven in \autoref{fig:Overzicht van meetwaarden bij verschillende Rout waarden}.
\begin{table}[h!]
\centering
\begin{tabular}{|c|c|c|c|c|c|c|c|}
\hline
\textbf{Rout} & \textbf{Vin} & \textbf{Iin} & \textbf{Vout} & \textbf{Iout} & \textbf{Pin} & \textbf{Pout} & \textbf{n} \\
\hline
2  & 48 & 6,011   & 11,8  & 5,9     & 288,528  & 69,62     & 24,13\% \\
4  & 48 & 3,245   & 11,817 & 2,954   & 155,76   & 34,907418 & 22,41\% \\
6  & 48 & 2,119   & 11,873 & 1,979   & 101,712  & 23,496667 & 23,10\% \\
8  & 48 & 1,632   & 11,901 & 1,488   & 78,336   & 17,708688 & 22,61\% \\
10 & 48 & 1,483   & 11,864 & 1,186   & 71,184   & 14,070704 & 19,77\% \\
12 & 48 & 1,286   & 11,866 & 0,999827 & 61,728   & 11,86394718 & 19,22\% \\
14 & 48 & 1,145   & 11,87  & 0,847829 & 54,96    & 10,06373023 & 18,31\% \\
16 & 48 & 1,039   & 11,875 & 0,742163 & 49,872   & 8,813185625 & 17,67\% \\
18 & 48 & 0,955579 & 11,875 & 0,659737 & 45,867792 & 7,834376875 & 17,08\% \\
20 & 48 & 0,74917  & 11,934 & 0,596704 & 35,96016  & 7,121065536 & 19,80\% \\
\hline
\end{tabular}
\caption{Overzicht van meetwaarden bij verschillende Rout waarden}\label{fig:Overzicht van meetwaarden bij verschillende Rout waarden}
\end{table}
\subsection{Inductor current ripple}
Voor de volgende opdracht krijgen wij een aantal waardes(\autoref{fig:Componentwaarden voor simulatie2.1}) gegeven, waarna we de waarde van de spoel zullen variëren om de effecten te observeren.
\begin{table}[h!]
\centering
\begin{tabular}{|l|l|}
\hline
\textbf{Component} & \textbf{Value} \\
\hline
Vin  & 48 volt \\
L    & 100 $\mu$H \\
C    & 10 $\mu$F \\
Rout & 10 \\
Fs   & 50 kHz \\
d    & 40\% \\
\hline
\end{tabular}
\caption{Componentwaarden voor simulatie}
\label{fig:Componentwaarden voor simulatie2.1}
\end{table}
De simulatie resulteert in de volgende waarden, zoals te zien in \autoref{fig:Inductor Current Ripple Waarden}
\begin{table}[h!]
\centering
\begin{tabular}{|l|c|c|c|}
\hline
\textbf{Inductor [H]} & \textbf{Current ripple [A]} & \textbf{MIN [A]} & \textbf{MAX [A]} \\
\hline
47u  & 2,948  & 0     & 2,948 \\
100u & 1,48   & 0,561 & 2,041 \\
150u & 1,577  & 0,056 & 1,633 \\
220u & 0,871  & 0,7   & 1,571 \\
470u & 0,313  & 1,053 & 1,366 \\
\hline
\end{tabular}
\caption{Inductor Current Ripple Waarden}
\label{fig:Inductor Current Ripple Waarden}
\end{table}
Dit herhalen we vervolgens, maar we veranderen de frequentie van 50 kHz naar 100 kHz. De resultaten zijn te zien in \autoref{fig:Inductor Current Ripple Waarden2}
\begin{table}[h!]
\centering
\begin{tabular}{|l|c|c|c|}
\hline
\textbf{Inductor [H]} & \textbf{Current ripple [A]} & \textbf{MIN [A]} & \textbf{MAX [A]}\\
\hline
47u  & 0,893067  & 0,716933  & 1,61 \\
100u & 0,892986  & 0,717014  & 1,61 \\
150u & 0,597264  & 0,869736  & 1,467 \\
220u & 0,28694   & 0,97206   & 1,259 \\
470u & 0,189     & 1,085     & 1,274 \\
\hline
\end{tabular}
\caption{Inductor Current Ripple Waarden}
\label{fig:Inductor Current Ripple Waarden2}
\end{table}
\subsection{Output voltage ripple}
\begin{table}[h!]
\centering
\begin{tabular}{|l|l|}
\hline
\textbf{Component} & \textbf{Value} \\
\hline
Vin  & 48 volt \\
L    & 100 $\mu$H \\
C    & 10 $\mu$F \\
Rout & 10 \\
Fs   & 50 kHz \\
d    & 40\% \\
\hline
\end{tabular}
\caption{Parameters for Output Voltage Ripple Simulation}
\label{fig:Output_voltage_ripple_simulation}
\end{table}
Als we deze gegevens invullen in de simulatie krijgen we de volgende resultaten te zien in \autoref{fig:Capacitor_voltage_ripple_waarden}
\begin{table}[h!]
\centering
\begin{tabular}{|l|c|c|c|}
\hline
\textbf{Capacitor [F]} & \textbf{Voltage ripple [V]} & \textbf{MIN [V]} & \textbf{MAX [V]} \\
\hline
10u   & 0,392 & 18,852 & 19,244 \\
22u   & 0,261 & 18,859 & 19,12  \\
47u   & 0,409 & 18,793 & 19,202 \\
100u  & 0,6   & 18,638 & 19,238 \\
220u  & 0,697 & 18,67  & 19,367 \\
470u  & 0,464 & 18,783 & 19,247 \\
\hline
\end{tabular}
\caption{Capacitor Voltage Ripple Waarden}
\label{fig:Capacitor_voltage_ripple_waarden}
\end{table}

Dit herhalen we vervolgens, maar we veranderen de frequentie van 50 kHz naar 100 kHz. De resultaten zijn te zien in \autoref{fig:Capacitor_voltage_ripple_waarden_100kHz}
\begin{table}[h!]
\centering
\begin{tabular}{|l|c|c|c|}
\hline
\textbf{Capacitor [F]} & \textbf{Voltage ripple [V]} & \textbf{MIN [V]} & \textbf{MAX [V]} \\
\hline
10u   & 0,087 & 18,962 & 19,049 \\
22u   & 0,012 & 18,995 & 19,007 \\
47u   & 0,269 & 18,88  & 19,149 \\
100u  & 0,299 & 18,858 & 19,157 \\
220u  & 1,018 & 18,434 & 19,452 \\
470u  & 0,699 & 18,609 & 19,308 \\
\hline
\end{tabular}
\caption{Capacitor Voltage Ripple Waarden bij 100 kHz}
\label{fig:Capacitor_voltage_ripple_waarden_100kHz}
\end{table}
\subsection{Output capacitor ESR}
In deze opdracht onderzoeken we de afhankelijkheid van de "output voltage ripple" van de equivalente seriesweerstand (Resr) van de condensator. We meten de piek-piek output voltage ripple bij verschillende Resr-waarden. De parameters voor de simulatie zijn als volgt:
\begin{table}[h!]
\centering
\begin{tabular}{|l|l|}
\hline
\textbf{Component} & \textbf{Value} \\
\hline
Vin  & 48 volt \\
L    & 100 $\mu$H \\
C    & 10 $\mu$F \\
Resr & 100 m$\Omega$ \\
Rout & 10 \\
Fs   & 50 kHz \\
d    & 40\% \\
\hline
\end{tabular}
\caption{Componentwaarden voor simulatie van de output voltage ripple}
\label{fig:Componentwaarden voor simulatie2.4}
\end{table}

De resultaten van de simulatie, waarin de Resr-waarde varieert, zijn te zien in \autoref{fig:Resr_Output_Voltage_Ripple_Waarden}.
\begin{table}[h!]
\centering
\begin{tabular}{|l|c|c|c|}
\hline
\textbf{Resr [\(\Omega\)]} & \textbf{Output voltage ripple [V]} & \textbf{MIN [V]} & \textbf{MAX [V]} \\
\hline
0,001  & 0,621  & 18,663  & 19,284 \\
0,01   & 0,583  & 18,713  & 19,296 \\
0,1    & 0,568  & 18,667  & 19,235 \\
0,2    & 0,578  & 18,703  & 19,281 \\
0,5    & 0,61   & 18,683  & 19,293 \\
1      & 0,562  & 18,667  & 19,229 \\
\hline
\end{tabular}
\caption{Output Voltage Ripple afhankelijk van Resr}
\label{fig:Resr_Output_Voltage_Ripple_Waarden}
\end{table}

Vervolgens verhogen we de frequentie van 50 kHz naar 100 kHz. De resultaten van deze simulatie zijn weergegeven in \autoref{fig:Resr_Output_Voltage_Ripple_Waarden_100kHz}.
\begin{table}[h!]
\centering
\begin{tabular}{|l|c|c|c|}
\hline
\textbf{Resr [\(\Omega\)]} & \textbf{Output voltage ripple [V]} & \textbf{MIN [V]} & \textbf{MAX [V]} \\
\hline
0,001  & 0,621  & 18,663  & 19,284 \\
0,01   & 0,583  & 18,713  & 19,296 \\
0,1    & 0,568  & 18,667  & 19,235 \\
0,2    & 0,578  & 18,703  & 19,281 \\
0,5    & 0,61   & 18,683  & 19,293 \\
1      & 0,562  & 18,667  & 19,229 \\
\hline
\end{tabular}
\caption{Output Voltage Ripple afhankelijk van Resr bij 100 kHz}
\label{fig:Resr_Output_Voltage_Ripple_Waarden_100kHz}
\end{table}
\subsection{Output capacitor ESR reduction}
In deze simulatie onderzoeken we de vermindering van de output voltage ripple door twee output condensatoren parallel te schakelen. De parameters van de simulatie zijn als volgt:
\begin{table}[h!]
\centering
\begin{tabular}{|l|l|}
\hline
\textbf{Component} & \textbf{Value} \\
\hline
Vin  & 48 volt \\
L    & 100 $\mu$H \\
C(2*) & 10 $\mu$F \\
Resr & 1 $\Omega$ \\
Rout & 10 \\
Fs   & 50 kHz \\
d    & 40\% \\
\hline
\end{tabular}
\caption{Componentwaarden voor simulatie van de output voltage ripple met parallelle condensatoren}
\label{fig:Componentwaarden_parallel_condensatoren}
\end{table}

De resultaten van de simulatie zijn als volgt weergegeven in \autoref{fig:Capacitor_output_ripple_parallel}.
\begin{table}[h!]
\centering
\begin{tabular}{|l|c|c|c|}
\hline
\textbf{Capacitor} & \textbf{Output voltage ripple} & \textbf{MIN} & \textbf{MAX} \\
\hline
2*10u   & 0,472  & 18,73  & 19,202 \\
1*22u   & 0,422  & 18,796 & 19,218 \\
\hline
\end{tabular}
\caption{Meting van de output voltage ripple bij parallelle condensatoren}
\label{fig:Capacitor_output_ripple_parallel}
\end{table}

\subsubsection{Conclusie}
De vermindering van de output voltage ripple bij parallelle condensatoren kan als volgt worden verklaard:

\textbf{Grotere totale capaciteit}: Bij parallelle schakeling worden de capaciteiten opgeteld, wat zorgt voor een grotere capaciteit. Dit verbetert de filtering en vermindert de spanningsrimpel.

\textbf{Lagere effectieve serieweerstand (ESR)}: Parallelle condensatoren delen de stroom, waardoor de totale ESR afneemt. Een lagere ESR betekent minder verliezen en dus minder rimpel.

\textbf{Betere spanningsafvlakking}: Door de toename in capaciteit en lagere ESR kunnen de condensatoren beter reageren op veranderingen, waardoor de rimpel in de uitgangsspanning verder afneemt.
\subsection{Waveforms Buck Converter, continuous inductor current}

\subsection{Waveforms Buck Converter, discontinuous inductor current}
\section{UC3842 IC as current mode controller}
\subsection{Supply voltage}
\begin{figure}[!h]
    \centering
    \includegraphics[width=0.7\linewidth]{img//hfd3/image.png}
    \caption{Breadboard}
    \label{fig:IDoNotAssociateWithNiggers}
\end{figure}







\subsection{Reference voltage}

\begin{table}[h!]
\centering
\begin{tabular}{|c|c|}
\hline
\textbf{Supply Voltage \(V_{supply}\)} & \textbf{Reference Voltage \(V_{ref}\)} \\ \hline
0 & 0 \\ \hline
5 & 0 \\ \hline
10 & 0 \\ \hline
15 & 0 \\ \hline
16 & 5 \\ \hline
17 & 5 \\ \hline
18 & 5 \\ \hline
19 & 5 \\ \hline
20 & 5 \\ \hline
\end{tabular}
\caption{\(V_{ref}\) als functie van \(V_{supply}\) voor toenemende voedings spanning}
\label{tab:toenemende_spanning_uc3842}
\end{table}

\begin{table}[h!]
\centering
\begin{tabular}{|c|c|}
\hline
\textbf{Supply Voltage \(V_{supply}\)} & \textbf{Reference Voltage \(V_{ref}\)} \\ \hline
20 & 5 \\ \hline
19 & 5 \\ \hline
18 & 5 \\ \hline
17 & 5 \\ \hline
16 & 5 \\ \hline
15 & 5 \\ \hline
14 & 5 \\ \hline
13 & 5 \\ \hline
12 & 5 \\ \hline
10 & 0 \\ \hline
5 & 0 \\ \hline
0 & 0 \\ \hline
\end{tabular}
\caption{\(V_{ref}\) als functie van \(V_{supply}\) voor afnemende voedings spanning}
\label{tab:afnemende_spanning_uc3842}
\end{table}

Bij het opvoeren van de spanning \(V_{supply}\) bij 16V wordt de spanning van \(V_{ref}\) 5V. En bij het afnemen van \(V_{supply}\) met een waarde van minder dan 10V, valt de spanning van \(V_{ref}\) van 5V naar 0V.

\begin{table}[h!]
\centering
\begin{tabular}{|l|c|}
\hline
\textbf{Parameter} & \textbf{Waarde (V)} \\ \hline
\(V_{ref}\) & 5 \\ \hline
\(V_{Supply}^{afnemend}\) & 10 \\ \hline
\(V_{Supply}^{toenemend}\) & 16 \\ \hline
\end{tabular}
\caption{Gemeten spanningen bij alleen NPN-transistor}
\label{tab:npn_waarden}
\end{table}
\subsection{Internal oscillator}
\textcolor{gray}{Wat is de minimum and maximum value of the voltage \(V_{osc}\)}. 

\(V_{osc}^{Max}\) = 3,3V 

\(V_{osc}^{Min}\) = 1,46V \newline \newline
Zie ook het scopebeeld \autoref{fig:min_max_vosc}.


\begin{figure}[!h]
    \centering
    \includegraphics[width=0.5\linewidth]{img//hfd3/TEK00008.png}
    \caption{\(V_{osc}\) Scopebeeld met min en max aangetoond}
    \label{fig:min_max_vosc}
\end{figure}
\subsection{Timing Capacitor and Resistor}

\subsection{Variabele frequentie}
\begin{table}[h!]
\centering
\begin{tabular}{|c|c|}
\hline
\textbf{Variabele weerstand (ohm)} & \textbf{Oscillator frequentie (kHz)} \\ \hline
0 & 97 \\ \hline
5k & 67 \\ \hline
20k & 49 \\ \hline
50k & 26 \\ \hline
75k & 15 \\ \hline
100k & 15 \\ \hline
\end{tabular}
\caption{Waarden van variabele weerstand en overeenkomstige oscillatorfrequenties}
\label{tab:oscillator_frequentie}
\end{table}

\subsection{Variable dutycycle}

\subsubsection{ISense}
\textcolor{gray}{Meet \( V_{\text{osc}} \) en \(V_{out}\). Wat is de maximale duty cycle?}

Zonder \(I_{sense}\) aan te sluiten hebben we een duty cycle van ongeveer 99\% welke wordt gegeven door de waarde van \(C_T\).

\paragraph{NPN}
In \autoref{fig:Scopebeeld_NPN_1} is te zien dat de duty cycle bijna 100\% is. Met de maximale en minimale waarden aangegeven in \autoref{tab:npn_waarden2}. Maar ook te zien in deze tabel is dat de spanning over \(V_{Isense}\) met 0.7V daalt door de spanningsval over \(V_{be}\). Daarom voegen we een pnp transistor toe die de \(V_{osc}\) buffert. Dit wordt uitgewerkt in \autoref{par:pnp_npn_samen}.
 
\begin{figure}[b]
    \centering
    \includegraphics[width=0.7\linewidth]{img/hfd3/TEK00016.PNG}
    \caption{Scopebeeld van \(V_{osc}\) (licht blauw) en \(V_{Isense}\) (donker blauw)}
    \label{fig:Scopebeeld_NPN_1}
\end{figure}


\begin{figure}[b]
    \centering
    \includegraphics[width=0.7\linewidth]{img/hfd3/TEK00048.PNG}
    \caption{Scopebeeld van \(V_{out}\) (licht blauw) en \(V_{osc}\) (donker blauw)}
    \label{fig:scopebeeld_pnp_2}
\end{figure}


\begin{table}[h!]
\centering
\begin{tabular}{|l|c|}
\hline
\textbf{Parameter} & \textbf{Waarde (V)} \\ \hline
Max Vosc & 3,09 \\ \hline
Min Vosc & 1,43 \\ \hline
Max VIsense & 2,51 \\ \hline
Min VIsense & 0,89 \\ \hline
\end{tabular}
\caption{Gemeten spanningen bij alleen NPN-transistor}
\label{tab:npn_waarden2}
\end{table}


\paragraph{PNP}
\label{par:pnp_npn_samen}
Een PNP transistor wordt toegevoegd om \(V_{osc}\) te bufferen en door de spanningsval over \(V_{be}\) gaat de spanning over \(V_{osc}\) met 0.7V omhoog. Dit is te zien in \autoref{tab:pnp_npn_waarden}, ook in \autoref{fig:scopebeeld_pnp_2} zien we de dubbele buffer.

\begin{table}[h!]
\centering
\begin{tabular}{|l|c|}
\hline
\textbf{Parameter} & \textbf{Vosc (V)} \\ \hline
Max VIsense & 3,16 \\ \hline
Min VIsense & 1,4 \\ \hline
\end{tabular}
\caption{Gemeten spanningen bij PNP en NPN-transistor}
\label{tab:pnp_npn_waarden}
\end{table}

\section{Voltage Mode Buck Converter}
\section{Bouwen en praktijkmetingen}
\subsection{Inleiding bouw van de PCB}
\begin{figure}[!h]
    \centering
    \includegraphics[width=0.5\linewidth]{img//hfd5/WhatsApp Image 2024-10-27 at 23.40.52_4fb77a8e.jpg}
    \caption{Overview}
    \label{fig:piemels}
\end{figure}
\subsection{Solderen PCB}

\begin{figure}[!h]
    \centering
    \includegraphics[width=0.5\linewidth]{img//hfd5/WhatsApp Image 2024-10-27 at 23.40.52_0f53b54c.jpg}
    \caption{Sexy overview}
    \label{fig:zwartepiemels}
\end{figure}
\section{Metingen Boost Converter}
\subsection{Open loop verkenning}
\subsubsection{Uitgangsspanning\&stroom}
\begin{table}[h!]
\centering
\begin{tabular}{|c|c|c|c|}
\hline
\textbf{Duty Cycle (\%)} & \textbf{Load Resistance (ohm)} & \textbf{Output Voltage (V)} & \textbf{Output Current (A)} \\ \hline
10 & 500 & 22.8 & 0.045 \\ \hline
20 & 500 & 29.3 & 0.058 \\ \hline
30 & 500 & 31.7 & 0.063 \\ \hline
40 & 500 & 33.6 & 0.067 \\ \hline
50 & 500 & 40.0 & 0.080 \\ \hline
60 & 500 & 49.4 & 0.098 \\ \hline
70 & 500 & 62.3 & 0.124 \\ \hline
80 & 500 & 59.0 & 0.115 \\ \hline
90 & 500 & 22.7 & 0.044 \\ \hline
\end{tabular}
\caption{Output voltage and current at different duty cycles with a load resistance of 500 ohm}
\label{tab:duty_cycle_output}
\end{table}

\begin{table}[h!]
\centering
\begin{tabular}{|c|c|c|c|}
\hline
\textbf{Duty Cycle (\%)} & \textbf{Load Resistance (ohm)} & \textbf{Output Voltage (V)} & \textbf{Output Current (A)} \\ \hline
10 & 50 & 21.2 & 0.419 \\ \hline
20 & 50 & 23.6 & 0.468 \\ \hline
30 & 50 & 26.6 & 0.526 \\ \hline
40 & 50 & 30.0 & 0.592 \\ \hline
50 & 50 & 27.0 & 0.534 \\ \hline
60 & 50 & 19.9 & 0.397 \\ \hline
70 & 50 & 13.6 & 0.270 \\ \hline
80 & 50 & 8.0 & 0.160 \\ \hline
90 & 50 & 3.2 & 0.063 \\ \hline
\end{tabular}
\caption{Output voltage and current at different duty cycles with a load resistance of 50 ohm}
\label{tab:duty_cycle_output_50ohm}
\end{table}
\subsubsection{Onbelaste meting}
\begin{table}[h!]
\centering
\begin{tabular}{|c|c|c|c|c|}
\hline
\textbf{Duty Cycle (\%)} & \textbf{Load Resistance (ohm)} & \textbf{Output Voltage (V)} & \textbf{Time (ms)} & \textbf{Scope Figure} \\ \hline
50 & 500 & 40.0 & 20 & A \\ \hline
50 & 50 & 26.9 & 55 & B \\ \hline
1 & 0 & - & - & C \\ \hline
\end{tabular}
\caption{Output voltage, time, and scope figure at various duty cycles and load resistances}
\label{tab:duty_cycle_scope}
\end{table}


\subsection{Rendement bij 560 Ohm}
\begin{table}[h!]
\centering
\begin{tabular}{|l|c|}
\hline
\textbf{Parameter} & \textbf{Value} \\ \hline
Input Voltage (V) & 19.9 / 7.761 \\ \hline
Input Current (A) & 0.39 \\ \hline
Output Voltage (V) & 19.6 / 0.7252 \\ \hline
Output Current (A) & 0.037 \\ \hline
Efficiency (\%) & 9\% \\ \hline
\end{tabular}
\caption{Parameter values for input and output measurements}
\label{tab:parameter_values}
\end{table}

\subsection{Spanning- en stroomvormen bij 560 Ohm}
\begin{equation}
U = 1.58 \, \text{V} \quad I = \frac{U}{R_3 \parallel R_4} = \frac{1.58}{1} = 1.58 \, \text{A}
\label{eq:formule 6.3}
\end{equation}

% \subsection{Rendement bij 56 Ohm}

% \subsection{Spanning en stroomvormen bij 56 Ohm}

% \subsection{Closed loop bij 560 Ohm}

% \subsection{Closed loop bij 56 Ohm}
\section{Meting aan een zelf gewikkelde spoel}
\textcolor{gray}{Opmerking: alles wat in het grijs staat betreft de opdrachtvragen.}
\subsection{Primaire wikkeling}
\begin{enumerate}
    \item \textcolor{gray}{Maak met montagedraad 10 wikkelingen.}
    \item \textcolor{gray}{Zoek in de datasheet naar de waarde van \( A_L \).}
    \item \( A_L = 1170 \pm 25\% \, \text{nH} \)
    \item \textcolor{gray}{Bereken de waarde van de inductiviteit \( L_p \).}
    \\ \( L_p = N^2 \times A_L = 10^2 \times 1170 \, \text{nH} \)
    \item \( L_p = 117 \, \mu \text{H} \)
    \item \textcolor{gray}{Meet met de LCR-meter de inductiviteit \( L_p \) van de 10 wikkelingen.}
    \item Gemeten \( L_p = 222 \, \mu \text{H} \)
    \item \textcolor{gray}{Klopt de gemeten inductiviteit \( L_{\text{gemeten}} \) met de berekende waarde \( L_p \)?}
    \\ Nee, de gemeten \( L_p \) komt niet overeen met de berekende waarde.
\end{enumerate}

\subsection{Secundaire wikkeling}
\begin{enumerate}
    \item \textcolor{gray}{Maak met montagedraad 5 wikkelingen om dezelfde kern.}
    \item \textcolor{gray}{Zoek in de datasheet naar de waarde van \( A_L \).}
    \item \( A_L = 1170 \pm 25\% \, \text{nH} \)
    \item \textcolor{gray}{Bereken de waarde van de inductiviteit \( L_s \).}
    \\ \( L_s = N^2 \cdot A_L = 5^2 \cdot 1170 \, \text{nH} \)
    \item Berekende \( L_s = 29.25 \, \mu \text{H} \)
    \item \textcolor{gray}{Meet met de LCR-meter de inductiviteit \( L_s \) van de 5 wikkelingen.}
    \item Gemeten \( L_s = 63.6 \, \mu \text{H} \)
    \item \textcolor{gray}{Klopt de gemeten inductiviteit \( L_{\text{gemeten}} \) met de berekende waarde \( L_s \)?}
    \\ Nee, de gemeten \( L_s \) komt niet overeen met de berekende waarde.
\end{enumerate}

\subsection{Primaire impedantie}
\begin{enumerate}
    \item \textcolor{gray}{Bereken de reactantie voor de inductiviteit met 10 wikkelingen.}
    \\ Reactantie \( X_{L,p} = 2 \pi f L = 2 \cdot \pi \cdot 100 \, \text{kHz} \cdot 222 \, \mu \text{H} \)
    \item\( X_{L,p} = 2 \pi f L = 2 \cdot \pi \cdot 100 \, \text{kHz} \cdot 222 \, \mu \text{H} = 139.5 \, \Omega \)
    \item \textcolor{gray}{Soldeer een weerstand van \(10 \, \Omega\) in serie met deze wikkeling.}
    \item \textcolor{gray}{Stel de functiegenerator in op een sinusspanning van 5 volt piek-piek, \( f = 100 \, \text{kHz} \).}
    \item \textcolor{gray}{Sluit de functiegenerator aan op de serieschakeling van de spoel en weerstand.}
    \item \textcolor{red}{Let op dat je de aarde van de functiegenerator en scope aan dezelfde klem aansluit.}
    \item \textcolor{gray}{Meet met de scope de spanning over de weerstand van 10\(\Omega\).}
    \\ \( U_{R,\text{pp}} = 380 \, \text{mV} \)
    \item \textcolor{gray}{Wat is de piek-piek amplitude van de stroom door de serieschakeling van de wikkeling en R.}
    \\ \(I_{\text{pp}} = \frac{U}{R + X_{L}} = \frac{5}{10 + 139.5} = 33.45 \, \text{mA}\)
    \item \textcolor{gray}{Bereken de impedantie voor de serieschakeling van de inductiviteit met 10 wikkelingen en de weerstand van 10\(\Omega\).}
    \\ \(Z = \sqrt{R^2 + X_{L}^2} = \sqrt{10^2 + 139.5^2}\)
    \item \textcolor{gray}{Impedantie Z}
    \\ \(Z = 139.9 \Omega\) 
    \item \textcolor{gray}{Bereken de stroom als over deze impedantie \(Z \) een sinusvormige spanning van 5 volt piek-piek, 100kHz staat.}
    \\ \(I_{\text{pp}} = \frac{U}{Z} = \frac{5}{139.9}\)
    \item \textcolor{gray}{Stroom i(piek-piek)}
    \\ \(I_{\text{pp}} = 35.75 \, \text{mA}\)
    \item \textcolor{gray}{Klopt de gemeten stroom met de berekende stroom?}
    \\ De gemeten en berekende waarden van de stroom liggen dicht bij elkaar, maar komen niet exact overeen.
\end{enumerate}
\subsection{Secundaire impedantie}
\begin{enumerate}
    \item \textcolor{gray}{Bereken de reactantie voor de inductiviteit met 5 wikkelingen.}
    \\ Reactantie \( X_{L,s} = 2 \pi f L = 2 \times \pi \times 100 \, \text{kHz} \times 63.6 \, \mu \text{H} \)
    
    \item \textcolor{gray}{Reactantie Xl}:
    \\ \(X_{L,s} = 2 \pi f L = 2 \times \pi \times 100 \times 10^3 \times 63.6 \times 10^{-6} = 39.96 \, \Omega\)
\end{enumerate}

\subsection{Secundaire spanning}
\begin{enumerate}
    \item \textcolor{gray}{Stel de functiegenerator in op een sinusspanning van 5 volt piek-piek, \( f = 100 \, \text{kHz} \).}
    \item \textcolor{gray}{Sluit de functiegenerator aan op de serieschakeling van de spoel en weerstand.}
    \item \textcolor{gray}{Bereken de wikkelverhouding en de verwachte secundaire spanning.}
    \\ \(\text{Wikkelverhouding} \, (P:S) = 10:5 \quad \Rightarrow \quad U_s = U_p \times \frac{N_s}{N_p} = 5 \times \frac{5}{10}=2.5\)
    \item \textcolor{gray}{Verwachte secundaire spanning}
    \\ \( U_s = 2.5 \, \text{V} \).
    \item \textcolor{red}{Let op dat je de aarde van de functiegenerator en scope aan dezelfde klem aansluit!}
    \item \textcolor{gray}{Meet met de scope de spanning over de secundaire wikkelingen.}
    \item \textcolor{gray}{Gemeten secundaire spanning}
    \\ \( U_{\text{gemeten}} = 1.54 + 0.88 = 2.42 \, \text{V} \).
    \item \textcolor{gray}{Verklaar het verschil}
    \\ Het minimale verschil komt waarschijnlijk door de interne impedantie van de oscilloscoop.
\end{enumerate}

\subsection{Nullast/Kortsluitproef}
\begin{enumerate}
    \item Stel de functiegenerator in op een sinusspanning van 5 volt piek-piek, \( f = 100 \, \text{kHz} \).
    \item Sluit de functiegenerator aan op de serieschakeling van de spoel en weerstand.
    \item Let op dat je de aarde van de functiegenerator en scope aan dezelfde klem aansluit.
    \item Laat de secundaire wikkeling open.
    \item Meet met de scope de spanning over de weerstand van \(10 \, \Omega\) en bereken hieruit de stroom (piek-piek).
    
    \begin{enumerate}
        \item Meetwaarde: \( U_{R,\text{pp}} = 10 \, \text{mV} \)
        \item Bereken de no-load stroom:
        \[
        I_{\text{No Load}} = \frac{U_{R,\text{pp}}}{R} = \frac{10 \, \text{mV}}{10 \, \Omega} = 1 \, \text{mA}
        \]
    \end{enumerate}

    \item Sluit nu een weerstand van \(1000 \, \Omega\) aan op de secundaire wikkeling.
    \item Meet met de scope de spanning over de weerstand van \(10 \, \Omega\) en bereken hieruit de stroom (piek-piek).
    
    \begin{enumerate}
        \item Meetwaarde: \( U_{R,\text{pp}} = 832 \, \text{mV} \)
        \item Bereken de stroom bij \(1 \, \text{k} \Omega\):
        \[
        I_{1k} = \frac{U_{R,\text{pp}}}{R} = \frac{832 \, \text{mV}}{10 \, \Omega} = 83.2 \, \text{mA}
        \]
    \end{enumerate}

    \item Verklaar het verschil tussen \( I_{\text{No Load}} \) en \( I_{1k} \): Bij no-load kan de stroom niet door de secundaire wikkeling stromen omdat er geen gesloten stroomkring is. Bij \(1 \, \text{k} \Omega\) kan de stroom wel vloeien, omdat de secundaire kring gesloten is.

    \item Sluit de secundaire wikkeling nu kort (verbind beide draadjes).
    \item Meet met de scope de spanning over de weerstand van \(10 \, \Omega\) en bereken hieruit de stroom (piek-piek).
    \item Gemeten stroom \( I_{\text{Short-Circuit}} = [ ] \, \text{V} \).
    \item Verklaar het verschil tussen \( I_{\text{No Load}} \) en \( I_{\text{Short-Circuit}} \).
\end{enumerate}

% \subsection{Waveforms Buck Converter, discontinuous inductor current}
% \subsection{Snelheid van de responsie}

\include{information/Sources}
\end{document}
