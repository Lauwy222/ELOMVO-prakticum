\subsection{Output capacitor ESR reduction}
In deze simulatie onderzoeken we de vermindering van de output voltage ripple door twee output condensatoren parallel te schakelen. De parameters van de simulatie zijn als volgt:
\begin{table}[h!]
\centering
\begin{tabular}{|l|l|}
\hline
\textbf{Component} & \textbf{Value} \\
\hline
Vin  & 48 volt \\
L    & 100 $\mu$H \\
C(2*) & 10 $\mu$F \\
Resr & 1 $\Omega$ \\
Rout & 10 \\
Fs   & 50 kHz \\
d    & 40\% \\
\hline
\end{tabular}
\caption{Componentwaarden voor simulatie van de output voltage ripple met parallelle condensatoren}
\label{fig:Componentwaarden_parallel_condensatoren}
\end{table}

De resultaten van de simulatie zijn als volgt weergegeven in \autoref{fig:Capacitor_output_ripple_parallel}.
\begin{table}[h!]
\centering
\begin{tabular}{|l|c|c|c|}
\hline
\textbf{Capacitor} & \textbf{Output voltage ripple} & \textbf{MIN} & \textbf{MAX} \\
\hline
2*10u   & 0,472  & 18,73  & 19,202 \\
1*22u   & 0,422  & 18,796 & 19,218 \\
\hline
\end{tabular}
\caption{Meting van de output voltage ripple bij parallelle condensatoren}
\label{fig:Capacitor_output_ripple_parallel}
\end{table}

\subsubsection{Conclusie}
De vermindering van de output voltage ripple bij parallelle condensatoren kan als volgt worden verklaard:

\textbf{Grotere totale capaciteit}: Bij parallelle schakeling worden de capaciteiten opgeteld, wat zorgt voor een grotere capaciteit. Dit verbetert de filtering en vermindert de spanningsrimpel.

\textbf{Lagere effectieve serieweerstand (ESR)}: Parallelle condensatoren delen de stroom, waardoor de totale ESR afneemt. Een lagere ESR betekent minder verliezen en dus minder rimpel.

\textbf{Betere spanningsafvlakking}: Door de toename in capaciteit en lagere ESR kunnen de condensatoren beter reageren op veranderingen, waardoor de rimpel in de uitgangsspanning verder afneemt.