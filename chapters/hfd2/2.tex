\subsection{Inductor current ripple}
Voor de volgende opdracht krijgen wij een aantal waardes(\autoref{fig:Componentwaarden voor simulatie2.1}) gegeven, waarna we de waarde van de spoel zullen variëren om de effecten te observeren.
\begin{table}[h!]
\centering
\begin{tabular}{|l|l|}
\hline
\textbf{Component} & \textbf{Value} \\
\hline
Vin  & 48 volt \\
L    & 100 $\mu$H \\
C    & 10 $\mu$F \\
Rout & 10 \\
Fs   & 50 kHz \\
d    & 40\% \\
\hline
\end{tabular}
\caption{Componentwaarden voor simulatie}
\label{fig:Componentwaarden voor simulatie2.1}
\end{table}
De simulatie resulteert in de volgende waarden, zoals te zien in \autoref{fig:Inductor Current Ripple Waarden}
\begin{table}[h!]
\centering
\begin{tabular}{|l|c|c|c|}
\hline
\textbf{Inductor} & \textbf{Current ripple} & \textbf{MIN} & \textbf{MAX} \\
\hline
47u  & 2,948  & 0     & 2,948 \\
100u & 1,48   & 0,561 & 2,041 \\
150u & 1,577  & 0,056 & 1,633 \\
220u & 0,871  & 0,7   & 1,571 \\
470u & 0,313  & 1,053 & 1,366 \\
\hline
\end{tabular}
\caption{Inductor Current Ripple Waarden}
\label{fig:Inductor Current Ripple Waarden}
\end{table}
Dit herhalen we vervolgens, maar we veranderen de frequentie van 50 kHz naar 100 kHz. De resultaten zijn te zien in \autoref{fig:Inductor Current Ripple Waarden2}
\begin{table}[h!]
\centering
\begin{tabular}{|l|c|c|c|}
\hline
\textbf{Inductor} & \textbf{Current ripple} & \textbf{MIN} & \textbf{MAX} \\
\hline
47u  & 0,893067  & 0,716933  & 1,61 \\
100u & 0,892986  & 0,717014  & 1,61 \\
150u & 0,597264  & 0,869736  & 1,467 \\
220u & 0,28694   & 0,97206   & 1,259 \\
470u & 0,189     & 1,085     & 1,274 \\
\hline
\end{tabular}
\caption{Inductor Current Ripple Waarden}
\label{fig:Inductor Current Ripple Waarden2}
\end{table}