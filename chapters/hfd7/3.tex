\subsection{Primaire impedantie}
\begin{enumerate}
    \item \textcolor{gray}{Bereken de reactantie voor de inductiviteit met 10 wikkelingen.}
    \\ Reactantie \( X_{L,p} = 2 \pi f L = 2 \cdot \pi \cdot 100 \, \text{kHz} \cdot 222 \, \mu \text{H} \)
    \item\( X_{L,p} = 2 \pi f L = 2 \cdot \pi \cdot 100 \, \text{kHz} \cdot 222 \, \mu \text{H} = 139.5 \, \Omega \)
    \item \textcolor{gray}{Soldeer een weerstand van \(10 \, \Omega\) in serie met deze wikkeling.}
    \item \textcolor{gray}{Stel de functiegenerator in op een sinusspanning van 5 volt piek-piek, \( f = 100 \, \text{kHz} \).}
    \item \textcolor{gray}{Sluit de functiegenerator aan op de serieschakeling van de spoel en weerstand.}
    \item \textcolor{red}{Let op dat je de aarde van de functiegenerator en scope aan dezelfde klem aansluit.}
    \item \textcolor{gray}{Meet met de scope de spanning over de weerstand van 10\(\Omega\).}
    \\ \( U_{R,\text{pp}} = 380 \, \text{mV} \)
    \item \textcolor{gray}{Wat is de piek-piek amplitude van de stroom door de serieschakeling van de wikkeling en R.}
    \\ \(I_{\text{pp}} = \frac{U}{R + X_{L}} = \frac{5}{10 + 139.5} = 33.45 \, \text{mA}\)
    \item \textcolor{gray}{Bereken de impedantie voor de serieschakeling van de inductiviteit met 10 wikkelingen en de weerstand van 10\(\Omega\).}
    \\ \(Z = \sqrt{R^2 + X_{L}^2} = \sqrt{10^2 + 139.5^2}\)
    \item \textcolor{gray}{Impedantie Z}
    \\ \(Z = 139.9 \Omega\) 
    \item \textcolor{gray}{Bereken de stroom als over deze impedantie \(Z \) een sinusvormige spanning van 5 volt piek-piek, 100kHz staat.}
    \\ \(I_{\text{pp}} = \frac{U}{Z} = \frac{5}{139.9}\)
    \item \textcolor{gray}{Stroom i(piek-piek)}
    \\ \(I_{\text{pp}} = 35.75 \, \text{mA}\)
    \item \textcolor{gray}{Klopt de gemeten stroom met de berekende stroom?}
    \\ De gemeten en berekende waarden van de stroom liggen dicht bij elkaar, maar komen niet exact overeen.
\end{enumerate}