\subsection{Nullast/Kortsluitproef}
\begin{enumerate}
    \item Stel de functiegenerator in op een sinusspanning van 5 volt piek-piek, \( f = 100 \, \text{kHz} \).
    \item Sluit de functiegenerator aan op de serieschakeling van de spoel en weerstand.
    \item Let op dat je de aarde van de functiegenerator en scope aan dezelfde klem aansluit.
    \item Laat de secundaire wikkeling open.
    \item Meet met de scope de spanning over de weerstand van \(10 \, \Omega\) en bereken hieruit de stroom (piek-piek).
    
    \begin{enumerate}
        \item Meetwaarde: \( U_{R,\text{pp}} = 10 \, \text{mV} \)
        \item Bereken de no-load stroom:
        \[
        I_{\text{No Load}} = \frac{U_{R,\text{pp}}}{R} = \frac{10 \, \text{mV}}{10 \, \Omega} = 1 \, \text{mA}
        \]
    \end{enumerate}

    \item Sluit nu een weerstand van \(1000 \, \Omega\) aan op de secundaire wikkeling.
    \item Meet met de scope de spanning over de weerstand van \(10 \, \Omega\) en bereken hieruit de stroom (piek-piek).
    
    \begin{enumerate}
        \item Meetwaarde: \( U_{R,\text{pp}} = 832 \, \text{mV} \)
        \item Bereken de stroom bij \(1 \, \text{k} \Omega\):
        \[
        I_{1k} = \frac{U_{R,\text{pp}}}{R} = \frac{832 \, \text{mV}}{10 \, \Omega} = 83.2 \, \text{mA}
        \]
    \end{enumerate}

    \item Verklaar het verschil tussen \( I_{\text{No Load}} \) en \( I_{1k} \): Bij no-load kan de stroom niet door de secundaire wikkeling stromen omdat er geen gesloten stroomkring is. Bij \(1 \, \text{k} \Omega\) kan de stroom wel vloeien, omdat de secundaire kring gesloten is.

    \item Sluit de secundaire wikkeling nu kort (verbind beide draadjes).
    \item Meet met de scope de spanning over de weerstand van \(10 \, \Omega\) en bereken hieruit de stroom (piek-piek).
    \item Gemeten stroom \( I_{\text{Short-Circuit}} = [ ] \, \text{V} \).
    \item Verklaar het verschil tussen \( I_{\text{No Load}} \) en \( I_{\text{Short-Circuit}} \).
\end{enumerate}
