\subsection{Secundaire spanning}
\begin{enumerate}
    \item \textcolor{gray}{Stel de functiegenerator in op een sinusspanning van 5 volt piek-piek, \( f = 100 \, \text{kHz} \).}
    \item \textcolor{gray}{Sluit de functiegenerator aan op de serieschakeling van de spoel en weerstand.}
    \item \textcolor{gray}{Bereken de wikkelverhouding en de verwachte secundaire spanning.}
    \\ \(\text{Wikkelverhouding} \, (P:S) = 10:5 \quad \Rightarrow \quad U_s = U_p \times \frac{N_s}{N_p} = 5 \times \frac{5}{10}=2.5\)
    \item \textcolor{gray}{Verwachte secundaire spanning}
    \\ \( U_s = 2.5 \, \text{V} \).
    \item \textcolor{red}{Let op dat je de aarde van de functiegenerator en scope aan dezelfde klem aansluit!}
    \item \textcolor{gray}{Meet met de scope de spanning over de secundaire wikkelingen.}
    \item \textcolor{gray}{Gemeten secundaire spanning}
    \\ \( U_{\text{gemeten}} = 1.54 + 0.88 = 2.42 \, \text{V} \).
    \item \textcolor{gray}{Verklaar het verschil}
    \\ Het minimale verschil komt waarschijnlijk door de interne impedantie van de oscilloscoop.
\end{enumerate}
