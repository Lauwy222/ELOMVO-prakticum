\subsection{Snelheid van de responsie}

\textcolor{gray}{Meet de maximale peak van de spoelstroom en de minimale peak} \newline
Maximale spoelstroom, \(I_{L_{MAX}}\) = 1,377A \newline
Minimale spoelstroom, \(I_{L_{MIN}}\) = 0,551A\newline

\textcolor{gray}{Meet de tijd die nodig is voor de spoelstroom om van de minimale waarde naar de maximale
waarde te gaan.} \newline
De tijd om van minimale peak naar maximale peak te komen is 0,01 us \newline
\newline
\textcolor{gray}{Kan je uit deze drie meetgegevens di/dt afleiden?} \newline
Ja, dat kan. Door de verandering in spoelstroom te delen door de tijd van de peak-to-peak. \newline

\textcolor{gray}{Kan je de waarde van de spoel berekenen met behulp van de gemeten di/dt en de in- en
uitgangsspanningen van de buck converter? Is dit dan de waarde van 100uH?}komen de berekeningen uit op 125\(\mu\)H dus niet 100\(\mu\)H. Dit komt doordat de bij \autoref{sub:1.5_ind_bel} gekozen spoelwaarde (125\(\mu\)H) aangehouden is, dus komt de berekening wel uit op de juiste spoelwaarde. 
