\subsection{CASPOC Simulaties Buck-converter}
\subsubsection{Verschillende duty cycles}
\textcolor{gray}{Vul onderstaande tabel in door de theoretische waarde van Vout te berekenen en de andere waarden af te
lezen uit de grafieken. De peak waarden zijn de allerhoogste waarden die voorkomen. Ook
worden de tijdstippen gevraagd waarop de peak plaats vindt. De 5\% settling tijd van een
signaal die als stationaire waarde 100 heeft is de tijd die nodig is om het signaal tussen de
95 en 105 te laten komen. Het signaal moet ook tussen deze waarden blijven.
}
\begin{table}[h!]
\centering
\resizebox{\textwidth}{!}{%
\begin{tabular}{|l|c|c|c|c|c|c|c|c|}
\hline
\textbf{d} & \textbf{Uitgangsspanning} & \textbf{Pieksapnning} & \textbf{Tijd piekspanning (s)} & \textbf{Stationaire spanning} & \textbf{Piekstroom} & \textbf{Tijd piekstroom (s)} & \textbf{Stationaire stroom} & \textbf{Settling tijd (s)} \\ \hline
10\% & 2 & 2,516 & 326\(\mu\) & 1,646 & 1,565 & 161\(\mu\) & 0,351 & 1,740m \\ \hline
25\% & 5 & 8,237 & 308\(\mu\) & 4,478 & 4,927 & 144\(\mu\) & 0,804 & 1,456m \\ \hline
50\% & 10 & 17,574 & 311\(\mu\) & 9,721 & 10,188 & 149\(\mu\) & 1,576 & 1,021m \\ \hline
75\% & 15 & 27,072 & 313\(\mu\) & 15,053 & 15,308 & 154\(\mu\) & 1,627 & 0,865m \\ \hline
90\% & 18 & 32,744 & 314\(\mu\) & 18,157 & 18,231 & 157\(\mu\) & 1,906 & 0,797m \\ \hline
\end{tabular}%
}
\caption{Nummerieke analyse buck converter 1}
\end{table}