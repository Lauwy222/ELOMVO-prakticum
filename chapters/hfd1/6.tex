\subsection{Vermogen}
\textcolor{gray}{Meet de gemiddelde ingangsstroom}\newline
Gemiddelde Ingangsstroom, \(I_{in, gem}\) = -0,623A , \(V_{in}\) = 20V , \(P_{in}\) = 12,46W \newline

\textcolor{gray}{Meet de gemiddelde uitgangsstroom}\newline
Gemiddelde Uitgangsstroom, \(I_{uit, gem}\) = -1,008A , \(V_{in}\) =  10,08V , \(P_{in}\) = 10,16W

\textcolor{gray}{Is het ingangsvermogen gelijk aan het uitgangsvermogen?}\newline
Het ingangs- en uitgangsvermogen zijn niet gelijk aan elkaar. Bij een ideale converter zou dit wel het geval zijn. In een echte buck converter zijn er echter kleine verliezen door weerstand in de spoel, het schakelen, en verliezen in de MOSFET en diode. Deze verliezen zorgen ervoor dat \(P_{in}\) iets groter is dan \(P_{uit}\).  