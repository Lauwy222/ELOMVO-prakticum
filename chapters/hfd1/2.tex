\subsection{Simulatie}
\begin{figure}[h!]
    \centering
    \includegraphics[width=1\linewidth]{img/hfd1/Schema voor de simulatie van een boostconverter.png}
    \caption{Schema voor de simulatie van een boostconverter}
    \label{fig:Schema voor de simulatie van een boostconverter}
\end{figure}
\begin{table}[ht]
\centering
\resizebox{\textwidth}{!}{%
\begin{tabular}{|l|c|c|c|}
\hline
\textbf{Duty Cycle (\%)} & \textbf{Vuit (V) Rl=1MEG (onbelast)} & \textbf{Vuit (V) Rl=560 Ohm} & \textbf{Vuit (V) Rl=56 Ohm} \\ \hline
10 & 42.77 & 42.57 & 41.98 \\ \hline
20 & 48.41 & 48.01 & 47.26 \\ \hline
40 & 65.02 & 64.34 & 62.91 \\ \hline
50 & 77.95 & 77.23 & 75.25 \\ \hline
75 & 139.54 & 138.83 & 132.72 \\ \hline
\end{tabular}%
}
\caption{Uitgangsspanningen Vuit bij verschillende belasting en duty cycles}
\end{table}
Uit de simulaties blijkt dat de uitgangsspanning toeneemt met een hogere duty cycle, zoals verwacht bij een boostconverter. De belasting beïnvloedt de spanning: bij zwaardere belasting daalt de uitgangsspanning licht. De resultaten komen goed overeen met de theorie en tonen de correcte werking van de schakeling. De wet van Kirchhoff en \(UL = L \cdot \frac{dI}{dt}\) zijn duidelijk waarneembaar in de stroom- en spanningsgolfvormen.

\subsubsection{10\% dutycycle, 1 MEG ohm load}
\begin{figure}[ht]
    \centering
    \begin{subfigure}[b]{0.3\linewidth}
        \centering
        \includegraphics[width=\linewidth]{img/hfd1/hfd1-10pduty-1MEG-INDUCTOR.png}
        \caption{Spoelstroom}
        \label{fig:inductor}
    \end{subfigure}
    \hfill
    \begin{subfigure}[b]{0.3\linewidth}
        \centering
        \includegraphics[width=\linewidth]{img/hfd1/hfd1-10pduty-1MEG-DIODE.png}
        \caption{Diodestroom}
        \label{fig:diode}
    \end{subfigure}
    \hfill
    \begin{subfigure}[b]{0.3\linewidth}
        \centering
        \includegraphics[width=\linewidth]{img/hfd1/hfd1-10pduty-1MEG-DMOSFET.png}
        \caption{Mosfetstroom}
        \label{fig:mosfet}
    \end{subfigure}
    
    \caption{Vergelijking van stroom en spanning bij verschillende componenten met een weerstand van 1M\(\Omega\) en duty cycle van 10\%.}
    \label{fig:componenten}
\end{figure}




\subsubsection{40\% duty cycle, 560 ohm load}
\begin{figure}[h!]
    \centering
    \begin{subfigure}[b]{0.3\linewidth}
        \centering
        \includegraphics[width=\linewidth]{img/hfd1/hfd1-40pduty-560-INDUCTOR.png}
        \caption{Spoelstroom}
        \label{fig:inductor40}
    \end{subfigure}
    \hfill
    \begin{subfigure}[b]{0.3\linewidth}
        \centering
        \includegraphics[width=\linewidth]{img/hfd1/hfd1-40pduty-560-DIODE.png}
        \caption{Diodestroom}
        \label{fig:diode40}
    \end{subfigure}
    \hfill
    \begin{subfigure}[b]{0.3\linewidth}
        \centering
        \includegraphics[width=\linewidth]{img/hfd1/hfd1-40pduty-560-MOSFET.png}
        \caption{Mosfetstroom}
        \label{fig:mosfet40}
    \end{subfigure}
    
    \caption{Vergelijking van stroom en spanning bij verschillende componenten met een weerstand van 560\(\Omega\) en duty cycle van 40\%.}
    \label{fig:componenten40}
\end{figure}




\subsubsection{75\% duty cycle, 56 ohm load}

\begin{figure}[h!]
    \centering
    \begin{subfigure}[b]{0.3\linewidth}
        \centering
        \includegraphics[width=\linewidth]{img/hfd1/hfd1-75pduty-56-INDUCTOR.png}
        \caption{Spoelstroom}
        \label{fig:inductor75}
    \end{subfigure}
    \hfill
    \begin{subfigure}[b]{0.3\linewidth}
        \centering
        \includegraphics[width=\linewidth]{img/hfd1/hfd1-75pduty-56-DIODE.png}
        \caption{Diodestroom}
        \label{fig:diode75}
    \end{subfigure}
    \hfill
    \begin{subfigure}[b]{0.3\linewidth}
        \centering
        \includegraphics[width=\linewidth]{img/hfd1/hfd1-75pduty-56-DMOSFET.png}
        \caption{Mosfetstroom}
        \label{fig:mosfet75}
    \end{subfigure}
    
    \caption{Vergelijking van stroom en spanning bij verschillende componenten met een weerstand van 56\(\Omega\) en duty cycle van 75\%.}
    \label{fig:componenten75}
\end{figure}

De simulaties lieten zien dat de boostconverter zich gedraagt zoals verwacht. Het was duidelijk dat de uitgangsspanning stijgt als de duty cycle toeneemt, wat aantoont dat de converter goed werkt. De stroomwet van Kirchhoff is duidelijk te zien met de relatie tussen V en I. Kijkend naar de simulatie wordt ook \(U_L = L * \frac{dI}{dT}\) duidelijk. In de grafieken is ook de theorie van de lessen terug te zien. Zo is het effect van de duty cycle en de samenhang hiervan met de verschillende stromen goed terug te zien in de diode-, mosfet- en spoelstromen.