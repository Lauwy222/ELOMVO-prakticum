\subsection{Reference voltage}

\begin{table}[h!]
\centering
\begin{tabular}{|c|c|}
\hline
\textbf{Supply Voltage \(V_{supply}\)} & \textbf{Reference Voltage \(V_{ref}\)} \\ \hline
0 & 0 \\ \hline
5 & 0 \\ \hline
10 & 0 \\ \hline
15 & 0 \\ \hline
16 & 5 \\ \hline
17 & 5 \\ \hline
18 & 5 \\ \hline
19 & 5 \\ \hline
20 & 5 \\ \hline
\end{tabular}
\caption{\(V_{ref}\) als functie van \(V_{supply}\) voor toenemende voedings spanning}
\label{tab:toenemende_spanning_uc3842}
\end{table}

\begin{table}[h!]
\centering
\begin{tabular}{|c|c|}
\hline
\textbf{Supply Voltage \(V_{supply}\)} & \textbf{Reference Voltage \(V_{ref}\)} \\ \hline
20 & 5 \\ \hline
19 & 5 \\ \hline
18 & 5 \\ \hline
17 & 5 \\ \hline
16 & 5 \\ \hline
15 & 5 \\ \hline
14 & 5 \\ \hline
13 & 5 \\ \hline
12 & 5 \\ \hline
10 & 0 \\ \hline
5 & 0 \\ \hline
0 & 0 \\ \hline
\end{tabular}
\caption{\(V_{ref}\) als functie van \(V_{supply}\) voor afnemende voedings spanning}
\label{tab:afnemende_spanning_uc3842}
\end{table}

Bij het opvoeren van de spanning \(V_{supply}\) bij 16V wordt de spanning van \(V_{ref}\) 5V. En bij het afnemen van \(V_{supply}\) met een waarde van minder dan 10V, valt de spanning van \(V_{ref}\) van 5V naar 0V.

\begin{table}[h!]
\centering
\begin{tabular}{|l|c|}
\hline
\textbf{Parameter} & \textbf{Waarde (V)} \\ \hline
\(V_{ref}\) & 5 \\ \hline
\(V_{Supply}^{afnemend}\) & 10 \\ \hline
\(V_{Supply}^{toenemend}\) & 16 \\ \hline
\end{tabular}
\caption{Gemeten spanningen bij alleen NPN-transistor}
\label{tab:npn_waarden}
\end{table}