\subsection{Variabele frequentie}
De frequentie zou op basis van formule \ref{Eq_Fosc} en figuur \ref{Timing Capacitor and Resistor} .
Hieruit is te lezen dat bij het vergroten van de weerstand bij een constante condensator waarde de frequentie kleiner wordt.


Bij het bouwen op de breadboard wordt een variable weerstand toegevoegd die de totale timing weerstand kan aanpassen. Dit betekend voor een grotere variable weerstand de frequentie zou moeten dalen. Dit kunnen we halen uit tabel \ref{tab:oscillator_frequentie}. 

\begin{table}[h!]
\centering
\begin{tabular}{|c|c|}
\hline
\textbf{Variabele weerstand (ohm)} & \textbf{Oscillator frequentie (kHz)} \\ \hline
0 & 97 \\ \hline
5k & 67 \\ \hline
20k & 49 \\ \hline
50k & 26 \\ \hline
75k & 15 \\ \hline
100k & 15 \\ \hline
\end{tabular}
\caption{Waarden van variabele weerstand en overeenstaande oscillator-frequenties}
\label{tab:oscillator_frequentie}
\end{table}
